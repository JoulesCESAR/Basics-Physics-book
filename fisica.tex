\chapter{Física}

\textit{``Sería muy triste ser un átomo en el universo sin los físicos. Y los físicos están hechos de átomos. Un físico es la 
forma de un átomo de saber que hay átomos.''} \textbf{George Wald}
\vspace{1.0 cm}

\section{Física: Filosofía Natural}

La Física es parte de las Ciencias Naturales, ésta ciencia estudia a la energía, la materia, el tiempo y el espacio en toda su 
comportamiento, estructura, e interacciones. El término \textbf{Física} proviene del lat. physica, y este del griego 
$\varphi\upsilon\sigma\iota\kappa o\varsigma$, 'natural, relativo a la naturaleza'.\\

Una muy precisa definición de Física es la siguiente:

\begin{tcolorbox}
Ciencia que estudia las leyes fundamentales que rigen el Universo.
\end{tcolorbox}

Entiendiéndose como Universo el conjunto de cuerpos celestes y materia interestellar que se encuentra en el espacio. También otra 
precisa definición es la siguiente:

\begin{tcolorbox}
Ciencia natural que estudia los fenómenos físicos y lo que ocurre en la naturaleza, los componentes fundamentales del Universo, 
la 
energía, la materia, el espacio-tiempo y las interacciones fundamentales. La Física es una ciencia básica estrechamente vinculada 
con las matemáticas y la lógica en la formulación y cuantificación de sus principios.
\end{tcolorbox}

En esta ciencia se estudia aquellos fenómenos llamados físicos en los cuales la estructura química de las materia permanence 
inalterada. Ejemplos de fenómenos físicos: la ebullición del agua, la caída de los cuerpos, la transmición del calor, la 
evaporización de los líquidos, etc.\\

Esta disciplina incentiva competencias, métodos y una cultura científica que permiten comprender nuestro mundo físico y viviente, 
para luego actuar sobre él. Sus procesos cognitivos se han convertido en protagonistas del saber y hacer científico y tecnológico 
general, ayudando a conocer, teorizar, experimentar y evaluar actos dentro de diversos sistemas, clarificando causa y efecto en 
numerosos fenómenos. De esta manera, la física contribuye a la conservación y preservación de recursos, facilitando la toma de 
conciencia y la participación efectiva y sostenida de la sociedad en la resolución de sus propios problemas.\\

Y como podemos apreciar nos encontramos viviendo en un mundo tecnológico y científico, siendo la física parte fundamental del 
mundo, ya que esta abarca desde lo infinitamente pequeño, como es el caso de la física de las partículas, a lo infinitamente 
grande, como es el caso de la astrofísica. Es por eso que no debe ser extraño para nosotros que la física se encuentre presente 
en 
cada ámbito del progreso técnico y científico.\\

La Física trata de responder dos grandes preguntas: ¿Cómo funciona el Universo? y ¿Por qué funciona?. La respuesta o las 
respuesta 
a la primera pregunta se manifiesta mediante el surgimiento de las leyes físicas, que cuyo sueño científico es lograr llegar a 
una 
sóla ley de unificación, mientras que la segunda pregunta se ve responde mediante el reconocimiento de principios físicos que a 
su 
vez responde a la inevitabilidad de los fenómenos en el Universo.
